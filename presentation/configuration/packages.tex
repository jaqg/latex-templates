\usepackage{fontspec}
\usepackage{appendixnumberbeamer}
\usepackage[scale=2]{ccicons}

%
% --- FIGURAS ---
%
\usepackage{graphicx}
\usepackage{import}
\usepackage{pdfpages}
\usepackage{calc}
\usepackage{subcaption}
\usepackage{wrapfig}
\usepackage[margin=10pt,labelfont=bf]{caption}
\usepackage[bottom]{footmisc} % Para evitar que las figuras aparezcan por debajo del footnote
%
% --- VIDEOS Y MULTIMEDIA ---
%
\usepackage{multimedia}
\usepackage{animate}
% \usepackage{movie15}
%
% --- TABLAS ---
%
\usepackage{booktabs}
\newcommand{\ra}[1]{\renewcommand{\arraystretch}{#1}} % Spacing btween lines of table
\usepackage{longtable}
\usepackage{array}
\usepackage{xtab}
\usepackage{multirow}
\usepackage{colortab}
\usepackage{bigdelim}
%
% -- Listas y estilo ---
%
\usepackage{enumerate}
\usepackage{enumitem}
% Reduce slide margins by doing \Wider[#em]{text} for # number
\newcommand\Wider[2][3em]{%
\makebox[\linewidth][c]{%
  \begin{minipage}{\dimexpr\textwidth+#1\relax}
  \raggedright#2
  \end{minipage}%
  }%
}
%
% --- Matemáticas y física ---
%
\usepackage{amsmath}
\newcommand{\mat}[1]{\boldsymbol{\mathrm{#1}}}
\usepackage{mathtools}
\usepackage[thinc]{esdiff}
\usepackage{physics}
\usepackage{braket}
\usepackage{empheq}
% \newtcbox{\mymathbox}[1][]{%
%     nobeforeafter, math upper, tcbox raise base,
%     standard jigsaw, sharp corners, colframe=white!45!black,
%     colback=white!95!black, leftrule=4pt, rightrule=0pt, toprule=0pt, bottomrule=0pt,
%     #1}
%
% --- Química ---
%
\usepackage{chemformula,array}
\usepackage[version=4]{mhchem}
\newcommand{\celsius}{$^{\circ}$C }
%
% --- Referencias ---
%
\newcommand{\equatref}[1]{Ecuación~(\ref{#1})}
\newcommand{\eqsref}[1]{Ecuaciones~(\ref{#1})}
\newcommand{\tabref}[1]{Tabla~\ref{#1}}
\newcommand{\figref}[1]{Figura~\ref{#1}}
%
% --- Otros ---
%
\usepackage{rotating}
\usepackage{lipsum}
\usepackage{xcolor}
\usepackage{hyperref}
\usepackage{blindtext}
\newcommand{\incfig}[1]{%
        \def\svgwidth{\columnwidth}
            \import{./figures/}{#1.pdf_tex}
        }
%
% --- Código ---
%
\usepackage{verbatimbox} % Centrar código con verbatim
\usepackage{listings}
\definecolor{dkgreen}{rgb}{0,0.6,0}
\definecolor{gray}{rgb}{0.5,0.5,0.5}
\definecolor{mauve}{rgb}{0.58,0,0.82}
 \lstdefinestyle{myStyle}{
    frame=tb,
    captionpos=t,
    belowcaptionskip=0.5\baselineskip,
    breaklines=true,
    columns=flexible,
    % frame=none,
    % numbers=none,
    numbers=left,
    numbersep=5pt,
    numberstyle=\tiny\color{gray},
    basicstyle=\footnotesize\ttfamily,
    keywordstyle=\bfseries\color{green!40!black},
    commentstyle=\itshape\color{white!40!black},
    identifierstyle=\color{blue},
    stringstyle=\color{mauve},
    % backgroundcolor=\color{gray!10!white},
    backgroundcolor=\color{white},
    showspaces=false,
    showstringspaces=false,
    tabsize=4,
    showtabs=false
}
