%
% --- Formato de página ---
%
\usepackage[a4paper, top=2cm, bottom=2cm, left=2cm, right=2cm]{geometry}
\usepackage{lscape}
\usepackage{multicol}
\usepackage{fancyhdr}
% \pagestyle{fancy}
\usepackage{setspace}
% \singlespace      % Interlineado normal
% \onehalfspace       % Interlineado de 1.5
% \doublespace      % Interlineado doble
\spacing{1.3}     % Interlineado personalizado
% \renewcommand{\baselinestretch}{1.5} % Interlinieado de 1.5
\usepackage{afterpage}

\newcommand\blankpage{%
    \null
    \thispagestyle{empty}%
    \addtocounter{page}{-1}%
    \newpage}
% In the text before the blank page, insert the following code: \afterpage{\blankpage}
%
% --- Fuentes ---
%
%\usepackage[adobe-utopia]{mathdesign}
%\usepackage{mathspec}
\usepackage[many]{tcolorbox}   % for colored boxes
% \usepackage{setspace}   % for line spacing
%\usepackage{multicol}   % for multicolumns
%
% --- Redefinicion de entornos ---
%

%
% --- Copyright ---
%

%
% --- Figuras ---
%
\usepackage{graphicx}
\usepackage{import}
\usepackage{xifthen}
\pdfminorversion=7
\usepackage{pdfpages}
\usepackage{transparent}
\usepackage{xifthen}
\usepackage{calc}
% \usepackage{caption}
\usepackage{subcaption}
\usepackage{wrapfig}
\usepackage[margin=10pt,labelfont=bf]{caption}
\usepackage[bottom]{footmisc} % Para evitar que las figuras aparezcan por debajo del footnote
%
% --- Tablas ---
%
\usepackage{booktabs}
% \usepackage{tablefootnote}
\newcommand{\ra}[1]{\renewcommand{\arraystretch}{#1}} % Spacing btween lines of table
\usepackage{longtable}
\usepackage{array}
\usepackage{xtab}
\usepackage{multirow}
\usepackage{colortab}
\usepackage{bigdelim}
%
% -- Listas y estilo ---
%
\usepackage{enumerate}
\usepackage{enumitem}
%
% --- Matemáticas y física ---
%
\newcommand{\mat}[1]{\boldsymbol{\mathrm{#1}}}
% \decimalpoint
\usepackage{mathtools}
\usepackage{cancel} 
\usepackage[thinc]{esdiff}
\usepackage{physics}
\usepackage{braket}
\pdfsuppresswarningpagegroup=1
%
% --- Química ---
%
\usepackage{chemfig}
\usepackage{chemformula,array} % \ch{}
\usepackage[version=4]{mhchem}
\newcommand{\celsius}{$^{\circ}$C }
%
% --- Referencias ---
%
\usepackage[colorlinks=true,
linkcolor = blue,
urlcolor  = blue,
citecolor = black,
anchorcolor = blue]{hyperref}
\usepackage{cleveref}
\newcommand{\equatref}[1]{eq.~(\ref{#1})}
\newcommand{\eqsref}[1]{Equations~(\ref{#1})}
\newcommand{\tabref}[1]{Table~\ref{#1}}
\newcommand{\figref}[1]{Figure~\ref{#1}}
%
% --- Otros ---
%
\usepackage{rotating}
\usepackage{lipsum}
\usepackage{xcolor}
\usepackage[nottoc,numbib]{tocbibind}
\usepackage{blindtext}
\newcommand{\incfig}[1]{%
        \def\svgwidth{\columnwidth}
            \import{./figures/}{#1.pdf_tex}
        }
%
% --- Colores ---
%
\definecolor{blue1}{HTML}{5989cf}    % setting main color to be used
\definecolor{blue2}{HTML}{cde4ff}     % setting sub color to be used
\definecolor{mycyan}{HTML}{CBE5EA}     % setting sub color to be used
\definecolor{mycyan2}{HTML}{dcdde2}     % setting sub color to be used
\definecolor{mymint}{HTML}{DAF7A6}     % setting sub color to be used
%
% --- Cajas ---
%
\tcbuselibrary{many}   % many : loads the libraries skins, breakable, raster, hooks, theorems, fitting and xparse .
\newtcolorbox{bluebox}[2][]{
	tile,
	frame style={left color=red!75!black,right color=red!10!yellow},
	boxrule = 2pt,
    opacityfill = 1.0,
	arc is angular,
	fonttitle = \bfseries,
	title={#2}, #1
}
% Boxed equation
\usepackage{empheq}
\newtcbox{\mymathbox}[1][]{%
    nobeforeafter, math upper, tcbox raise base,
    standard jigsaw, sharp corners, colframe=white!45!black,
    colback=white!95!black, leftrule=4pt, rightrule=0pt, toprule=0pt, bottomrule=0pt,
    #1}
%
% --- Código ---
%
\usepackage{verbatimbox} % Centrar código con verbatim
%
% --- Dedicatoria ---
%
\newenvironment{dedication}
{
   \cleardoublepage
   \thispagestyle{empty}
   \vspace*{\stretch{1}}
   \hfill\begin{minipage}[t]{0.66\textwidth}
   \raggedright
}%
{
   \end{minipage}
   \vspace*{\stretch{3}}
   \clearpage
}

% \newenvironment{dedication}
%   {\clearpage           % we want a new page
%    \thispagestyle{empty}% no header and footer
%    \vspace*{\stretch{1}}% some space at the top
%    \itshape             % the text is in italics
%    \raggedleft          % flush to the right margin
%   }
%   {\par % end the paragraph
%    \vspace{\stretch{3}} % space at bottom is three times that at the top
%    \clearpage           % finish off the page
%   }
%Clears plain-page pg# settings, relocates pg#'s @ top-right-corner.
\makeatletter
\renewcommand{\ps@plain}{
\renewcommand\@oddhead{\hfill\normalfont\textrm{\thepage}}
\renewcommand\@evenhead{}
\renewcommand\@oddfoot{}
\renewcommand\@evenfoot{}}
%
% --- Codigo ---
%
% \usepackage{minted}
\usepackage{listings}
\definecolor{dkgreen}{rgb}{0,0.6,0}
\definecolor{gray}{rgb}{0.5,0.5,0.5}
\definecolor{mauve}{rgb}{0.58,0,0.82}

% \lstset{frame=tb,
%   language=Fortran,
%   aboveskip=3mm,
%   belowskip=3mm,
%   showstringspaces=false,
%   columns=flexible,
%   basicstyle={\small\ttfamily},
%   numbers=none,
%   numberstyle=\tiny\color{gray},
%   keywordstyle=\color{blue},
%   commentstyle=\color{dkgreen},
%   stringstyle=\color{mauve},
%   breaklines=true,
%   breakatwhitespace=true,
%   tabsize=3
% }
 \lstdefinestyle{myStyle}{
    frame=tb,
    captionpos=t,
    belowcaptionskip=0.5\baselineskip,
    breaklines=true,
    columns=flexible,
    % frame=none,
    % numbers=none,
    numbers=left,
    numbersep=5pt,
    numberstyle=\tiny\color{gray},
    basicstyle=\footnotesize\ttfamily,
    keywordstyle=\bfseries\color{green!40!black},
    commentstyle=\itshape\color{white!40!black},
    identifierstyle=\color{blue},
    stringstyle=\color{mauve},
    % backgroundcolor=\color{gray!10!white},
    backgroundcolor=\color{white},
    showspaces=false,
    showstringspaces=false,
    tabsize=4,
    showtabs=false
}
